\documentclass[11pt]{article}

\usepackage{amsmath,amsfonts,amssymb,mathtools,nicefrac,cases,empheq,enumitem}
\usepackage{xcolor,tikz,graphicx}

% Formatting
\usepackage[utf8]{inputenc}
\usepackage[top=1in, bottom=1in, right=1in, left=1in]{geometry} % page margins
\setlength{\parindent}{0 pt} % paragraph left indentation
\setlength{\parskip}{36 pt} % before paragraph spacing
\setlength{\jot}{12pt} % between-line spacing in multi-line equations
\renewcommand{\baselinestretch}{1.25} % between-line spacing
\usepackage{titlesec,float} % sections spacing + indent first paragraph in section, placing figures
\titleformat*{\section}{\large\bfseries}
\titlespacing*{\section}{0 pt}{0 pt}{-24 pt} % title spacing
\titlespacing*{\subsection}{0 pt}{0 pt}{0 pt}
\setlength{\abovecaptionskip}{6 pt} % moving figure captures up
%\setlength{\abovecaptionskip}{-12 pt} % reducing space below figures
\allowdisplaybreaks


\title{\vspace{-36 pt} \Large Perturbative solution for $\Lambda(f_0)$ in the limit that $\rho$ is small \vspace{-36 pt}}
\date{}

\begin{document}
% \maketitle
\section*{Perturbative solution of the generating function for small $f_0$}
We are interested in the linkage equilibrium statistic $\Lambda(f_0)$, defined as
\begin{align}
    \Lambda(f_0) \equiv \frac{\left\langle f_{ab}f_{Ab}f_{aB}f_{AB} \cdot e^{-\frac{f_{A}+f_{B}}{f_0}}\right\rangle}{\left\langle f_A^2(1-f_A)^2f_B^2(1-f_B)^2\cdot e^{-\frac{f_{A}+f_{B}}{f_0}}\right\rangle},
\end{align}
where $f_A \equiv f_{Ab} + f_{aB}$, $f_B \equiv f_{aB} + f_{AB}$ and $f_A, f_B \lesssim f_0$.
In the limit that $f_{Ab}$, $f_{aB}$, $f_{AB} \ll1$, 
\begin{align}\label{eq:lambda_small_f}
    \Lambda(f_0) \approx \frac{\left\langle f_{Ab}f_{aB}f_{AB} \cdot e^{-\frac{f_{A}+f_{B}}{f_0}}\right\rangle}{\left\langle f_A^2f_B^2\cdot e^{-\frac{f_{A}+f_{B}}{f_0}}\right\rangle}.
\end{align}
The moments follows from 
\begin{align}\label{eq:lambda_numerator}
    \left\langle f_{Ab}^if_{aB}^jf_{AB}^k\cdot e^{-\frac{f_{A}+f_{B}}{f_0}}\right\rangle
    = -f_0^{(i+j+k)} \partial_x^i \partial_y^j \partial_z^k H(x, y, z, t) \Bigg\vert_{\substack{x=1 \\ y=1 \\ z=2 \\ t=\infty}},
\end{align}
where %$m = i+j+k$ and 
\begin{align}
    H(x, y, z, t) \equiv \left\langle e^{-x\frac{f_{Ab}(t)}{f_0}-y\frac{f_{aB}(t)}{f_0}-z\frac{f_{AB}(t)}{f_0}} \right\rangle 
\end{align}
is the joint moment generating function. 
%The weighted moment in the numerator of Eq.(\ref{eq:lambda_small_f}) can be obtained as
%\begin{align}\label{eq:lambda_numerator}
%    \left\langle f_{Ab}f_{aB}f_{AB} \cdot e^{-\frac{f_{A}+f_{B}}{f_0}}\right\rangle
%    = -f_0^3 \frac{\partial^3}{\partial x \partial y \partial z} \lim_{t\to\infty} H(x+1, y+1, z+2, t).
%\end{align}
%and 
%\begin{align}\label{eq:lambda_denominator}
%    \left\langle f_{A}f_{B} \cdot e^{-\frac{f_{A}+f_{B}}{f_0}}\right\rangle &=
    %\left\langle f_{Ab}f_{aB} \cdot e^{-\frac{f_{A}+f_{B}}{f_0}}\right\rangle 
    %+ \left\langle f_{Ab}f_{AB} \cdot e^{-\frac{f_{A}+f_{B}}{f_0}}\right\rangle \\
    %&\quad + \left\langle f_{aB}f_{AB} \cdot e^{-\frac{f_{A}+f_{B}}{f_0}}\right\rangle 
    %+ \left\langle f_{AB}^2 \cdot e^{-\frac{f_{A}+f_{B}}{f_0}}\right\rangle \\
    %& \quad = 
%    f_0^2 \left[\frac{\partial^2}{\partial x \partial y} + \frac{\partial^2}{\partial x \partial z} + \frac{\partial^2}{\partial y \partial z} + \frac{\partial^2}{\partial z^2}\right] \lim_{t\to\infty} H(x+1, y+1, z+2, t).
%\end{align}


In the limit that $\theta = 2N\mu$ and $f_0$ are both small compared to one, 
\begin{align}\label{eq:h_expansion}
    H(x, y, z, \tau) &\approx 1 -\theta (H_A + H_B)
    + \frac{\theta^2}{2}\left(H_A + H_B\right)^2 \\\nonumber
    &\quad + \theta^2f_0\Upsilon + \theta^2f_0\int_0^{\tau}d \tau' z(\tau') \left[\Phi_x(\tau')+\Phi_y(\tau')-\rho\Phi_x(\tau')\Phi_y(\tau')\right] \\\nonumber
    &\quad + \mathcal{O}(f_0^2) +\mathcal{O}(\theta^3),
\end{align}
where $\tau = \nicefrac{t}{2Nf_0}$, $\gamma_A = 2Ns_Af_0$, $\gamma_B = 2Ns_Bf_0$, $\rho = 2NRf_0$, and 
\begin{subequations}\begin{align}
    H_A(x, \tau) &\equiv \ln \left[1 + \frac{x(1-e^{-\gamma_A\tau})}{\gamma_A}\right], \\
    H_B(y, \tau) &\equiv \ln \left[1 + \frac{y(1-e^{-\gamma_B\tau})}{\gamma_B}\right],
\end{align}\end{subequations}
\begin{subequations}\begin{align}
    \Phi_x(\tau') &\equiv -\frac{[ 1-e^{-\gamma_A (\tau-\tau')} ][\gamma_A+x(1-e^{-\gamma_A \tau'})]}{\gamma_A \left[ \gamma_A+x(1-e^{-\gamma_A \tau}) \right]}, \\
    \Phi_y(\tau') &\equiv -\frac{[1-e^{-\gamma_B (\tau-\tau')}][\gamma_B+y(1-e^{-\gamma_B \tau'})]}{\gamma_A \left[ \gamma_B+y(1-e^{-\gamma_B \tau}) \right]},
\end{align}\end{subequations}
\begin{align}
    \Upsilon(x, y, \tau) &= \int_0^{\tau} d\tau' \rho \left[x(\tau') + y(\tau')\right]\Phi_x(\tau')\Phi_y(\tau').
\end{align}

The characteristic $z(\tau')$ is defined by 
\begin{align}\label{eq:z_general_eq}
    \partial_{\tau'} z(\tau') = -(\gamma_{AB} + \rho) z(\tau') - z^2(\tau') + \rho \frac{\gamma_Axe^{-\gamma_A\tau'}}{\gamma_A+x(1-e^{-\gamma_A\tau'})} + \rho \frac{\gamma_Bye^{-\gamma_B\tau'}}{\gamma_B+y(1-e^{-\gamma_B\tau'})}
\end{align}
with the initial condition $z(0)=z$, where $\gamma_{AB}$ is the rescaled fitness of the double mutant.

\section*{Perturbative solution for $\Lambda(f_0)$ for neutral loci and weak recombination}
In the absence of recombination, the equation above has an exact solution, 
\begin{align}
    z_0(\tau') = \frac{\gamma_{AB}ze^{-\gamma_{AB}\tau'}}{\gamma_{AB}+z(1-e^{-\gamma_{AB}\tau'})}.
\end{align}
In the limit that $\rho \ll 1$, corrections to the zeroth-order solution can be found by perturbatively expanding $z(\tau')$ as
\begin{align}\label{eq:z_series}
    z(\tau') \approx z_0(\tau') + \sum_{i=1}^{\infty} \rho^{i}z_i(\tau').
\end{align}
Plugging the series expansion in the equation for $z(\tau')$ and matching the coefficients in front powers of $\rho$, we obtain for the first-order correction
\begin{align}
        \partial_{\tau'}z_1(\tau') \approx -\gamma_{AB}z_1(\tau') -2z_0(\tau')z_1(\tau') - z_0(\tau') + \frac{\gamma_Axe^{-\gamma_A\tau'}}{\gamma_A+x(1-e^{-\gamma_A\tau'})} + \frac{\gamma_Bye^{-\gamma_B\tau'}}{\gamma_B+y(1-e^{-\gamma_B\tau'})}.
\end{align}

In the neutral limit, the equation above reduces to
\begin{align}\label{eq:z_neutral_eq}
    \partial_{\tau'}z_1(\tau') &\approx - \frac{2z}{1+z\tau'}z_1(\tau') - \frac{z}{1+z\tau'} + \frac{x}{1+x\tau'} + \frac{y}{1+y\tau'}
\end{align}
with the initial condition $z_1(0)=0$. Using the method of variation of constants, we find
%This inhomogeneous linear ordinary differential equation can be solved by the method of variation of constants. The corresponding homogeneous equation 
%\begin{align}
%    \partial_{\tau'}z_1(\tau') &\approx - \frac{2z}{1+z\tau'}z_1(\tau')
%\end{align}
%has solution in the form
%\begin{align}\label{eq:z_1_homogeneous_u}
%    z_1(\tau') \approx \frac{\phi(\tau')}{(1+z\tau')^2},
%\end{align}
%where $\phi(\tau')$ is some function of $\tau'$. Plugging Eq. (\ref{eq:z_1_homogeneous_u}) into Eq. (\ref{eq:z_neutral_eq}), we obtain
%\begin{align}
%    \partial_{\tau'} \phi(\tau') \approx - z(1+z\tau') + \frac{x(1+z\tau')^2}{1+x\tau'} + \frac{y(1+z\tau')^2}{1+y\tau'},
%\end{align}
%from where
%\begin{align}
%    \phi(\tau') &\approx - \int z(1+z\tau') \,d\tau'\
%    + \int \frac{x(1+z\tau')^2}{1+x\tau'} \,d\tau'\
%    + \int \frac{y(1+z\tau')^2}{1+y\tau'} \,d\tau'\, \\\nonumber
%    &= \frac{1}{2} (1+z\tau')^2 + z\tau'\left(1 - \frac{z}{x}\right) + z\tau'\left(1 - \frac{z}{y}\right)
%     \\\nonumber
%    &\quad + \left(1-\frac{z}{x}\right)^2\ln(1+x\tau')
%    + \left(1-\frac{z}{y}\right)^2\ln(1+y\tau') + C,
%\end{align}
%where $C = \frac{1}{2}$ is a constant determined by the initial condition $\phi(0)=0$. Then, 
\begin{align}\label{eq:z_1}
    z_1(\tau') &\approx \frac{1}{2} 
    + \frac{1}{2}\frac{1}{(1+z\tau')^2} + \left(1-\frac{z}{x}\right)\frac{z\tau'}{(1+z\tau')^2} + \left(1-\frac{z}{y}\right)\frac{z\tau'}{(1+z\tau')^2}
         \\\nonumber
    &\quad + \left(1-\frac{z}{x}\right)^2 \frac{\ln(1+x\tau')}{(1+z\tau')^2}
    + \left(1-\frac{z}{y}\right)^2 \frac{\ln(1+y\tau')}{(1+z\tau')^2}.
\end{align}
%Substituting Eq. (\ref{eq:z_1}) into Eq. (\ref{eq:z_series}), we find
%\begin{align}
%    z(\tau') &\approx \frac{z}{1+z\tau'} 
%    + \frac{\rho}{2}
%    + \frac{\rho}{2}\frac{1}{(1+z\tau')^2} 
%    \\\nonumber
%    &\quad + \rho\left(1-\frac{z}{x}\right)\frac{z\tau'}{(1+z\tau')^2} + \rho\left(1-\frac{z}{y}\right)\frac{z\tau'}{(1+z\tau')^2}
%    \\\nonumber
%    &\quad + \rho\left(1-\frac{z}{x}\right)^2 \frac{\ln(1+x\tau')}{(1+z\tau')^2}
%    + \rho\left(1-\frac{z}{y}\right)^2 \frac{\ln(1+y\tau')}{(1+z\tau')^2}.
%\end{align}

Thus, to the lowest order in $\rho$, from Eq. (\ref{eq:lambda_numerator}) the numerator of $\Lambda(f_0)$ follows as
\begin{align}\label{eq:lambda_num_contribution}
    \left\langle f_{Ab}f_{aB}f_{AB}\cdot e^{-\frac{f_{A}+f_{B}}{f_0}}\right\rangle \approx&{} 
    -\theta^2 f_0^4 \int_0^{\tau} d\tau'\, \partial_x\partial_y\partial_z \left[-\rho z_0\Phi_x\Phi_y + \rho z_1\Phi_x + \rho z_1\Phi_y\right]\Bigg\vert_{\substack{x=1 \\ y=1 \\ z=2 \\ \tau=\infty}} \\ \nonumber
    &\approx \rho \theta^2 f_0^4 \left[
    \int_0^{\tau} d\tau'\, \partial_x \Phi_x \, \partial_y \Phi_y \, \partial_z \frac{z}{1+z\tau'} \right. \\\nonumber
    &+ \left. \int_0^{\tau} d\tau'\, \partial_x \Phi_x \, \partial_y \partial_z \frac{z}{y} \frac{z\tau'}{(1+z\tau')^2} \right. \\\nonumber
    &+ \left. \int_0^{\tau} d\tau'\, \partial_y \Phi_y \, \partial_x \partial_x \frac{z}{x} \frac{z\tau'}{(1+z\tau')^2} \right. \\\nonumber
    &- \left. \int_0^{\tau} d\tau'\, \partial_x \Phi_x \, \partial_y \partial_z \left(1-\frac{z}{y}\right)^2 \frac{\ln(1+y\tau')}{(1+z\tau')^2} \right. \\\nonumber
    &- \left. \int_0^{\tau} d\tau'\, \partial_y \Phi_y \, \partial_x \partial_z \left(1-\frac{z}{x}\right)^2 \frac{\ln(1+x\tau')}{(1+z\tau')^2} 
    \right]\Bigg\vert_{\substack{x=1 \\ y=1 \\ z=2 \\ \tau=\infty}} \\ \nonumber
    % &= -\rho \theta^2 f_0^4 \left[
    % -\int_0^{\infty} \frac{1}{(1+2\tau')^2}\,d\tau 
    % + 2 \int_0^{\infty} \frac{4\tau'}{(1+2\tau')^3} \,d\tau' \right. \\ \nonumber
    % &+ \left. 2 \int_0^{\tau} \frac{2\tau'-2(3+2\tau')\ln(1+\tau')}{(1+2\tau')^3} \,d\tau' \right] \\ \nonumber
    &= \rho \theta^2 f_0^4 \left[\frac{1}{2} -\frac{1}{2} -\frac{1}{2} +\frac{3}{4} +\frac{3}{4}\right] = \rho \theta^2 f_0^4.
\end{align}
where have used 
\begin{subequations}\begin{align}
    %\Phi_x(\tau') &= -\frac{(\tau-\tau')(1+x\tau')}{1+x\tau}
    \partial_x \Phi_x \Bigg\vert_{\substack{x=1 \\ \tau=\infty}} = 
    %-(\tau-\tau') \partial_x \frac{1+x\tau'}{1+x\tau} \Bigg\vert_{\substack{x=1 \\ \tau=\infty}} =
    \frac{(\tau-\tau')^2}{(1+x\tau)^2} \Bigg\vert_{\substack{x=1 \\ \tau=\infty}} = 1, \\
    %\Phi_y(\tau') &= -\frac{(\tau-\tau')(1+y\tau')}{1+y\tau}.
    \partial_y \Phi_y \Bigg\vert_{\substack{y=1 \\ \tau=\infty}} =
    %-(\tau-\tau') \partial_x \frac{1+x\tau'}{1+x\tau} \Bigg\vert_{\substack{x=1 \\ \tau=\infty}} =
    \frac{(\tau-\tau')^2}{(1+y\tau)^2} \Bigg\vert_{\substack{y=1 \\ \tau=\infty}} = 1.
\end{align}\end{subequations}

%Taking  in Eq. (\ref{eq:lambda_num_contribution}), we obtain 
%\begin{align}\label{eq:lambda_num_contribution_solved}
%    \left\langle f_{Ab}f_{aB}f_{AB}\cdot e^{-\frac{f_{A}+f_{B}}{f_0}}\right\rangle \approx& -\rho \theta^2 f_0^4 \left[
%    -\int_0^{\infty} \frac{1}{(1+2\tau')^2}\,d\tau  \right. \\ \nonumber
%    &+ \left. 2 \int_0^{\infty} \frac{4\tau'}{(1+2\tau')^3} \,d\tau' \right. \\ \nonumber
%    &+ \left. 2 \int_0^{\tau} \frac{2\tau'-2(3+2\tau')\ln(1+\tau')}{(1+2\tau')^3} \,d\tau' \right] \\ \nonumber
%    &= -\rho \theta^2 f_0^4 \left[-\frac{1}{2} + 1 - \frac{3}{2}\right] = \rho \theta^2 f_0^4.
%\end{align}

%The first integral in Eq. (\ref{eq:lambda_num_contribution}) evaluates to
%\begin{align}\label{eq:small_rho_num_1}
%    -\int_0^{\tau} d\tau'\, \partial_x \Phi_x \, \partial_y \Phi_y \, \partial_z \frac{z}{1+z\tau'}\Bigg\vert_{\substack{x=1 \\ y=1 \\ z=2 \\ \tau=\infty}} & = -\int_0^{\infty} \frac{1}{(1+2\tau')^2}\,d\tau = -\frac{1}{2}.
%\end{align}

%The second (and third) integral in Eq. (\ref{eq:lambda_num_contribution}) can be evaluated as \begin{align}\label{eq:small_rho_num_2}
%    & -\int_0^{\tau} d\tau'\, \partial_x \Phi_x \, \partial_y \partial_z \frac{z}{y} \frac{z\tau'}{(1+z\tau')^2} \Bigg\vert_{\substack{x=1 \\ y=1 \\ z=2 \\ \tau=\infty}} 
    %= -\int_0^{\tau} d\tau'\, \partial_y \Phi_y \, \partial_x \partial_z \frac{z}{x} \frac{z\tau'}{(1+z\tau')^2} \Bigg\vert_{\substack{x=1 \\ y=1 \\ z=2 \\ \tau=\infty}} 
    %= \int_0^{\infty} \frac{2z\tau'}{(1+z\tau')^3} \,d\tau'
%    = \int_0^{\infty} \frac{4\tau'}{(1+2\tau')^3} \,d\tau' = \frac{1}{2}.
%\end{align}

%Finally, we find the last (and the second to last) integral in Eq. (\ref{eq:lambda_num_contribution}) as
%\begin{align}\label{eq:small_rho_num_3}
%    \int_0^{\tau} d\tau'\, \partial_y \Phi_y \, \partial_x \partial_z \left(1-\frac{z}{x}\right)^2 \frac{\ln(1+x\tau')}{(1+z\tau')^2} \Bigg\vert_{\substack{x=1 \\ y=1 \\ z=2 \\ \tau=\infty}} &= %\int_0^{\tau} \left[-\frac{2\tau'(x-z)}{x^2(1+z\tau')^3}-\frac{2\ln(1+x\tau')}{x^3(1+z\tau')^3}\left[z(1+x\tau')-(x-z)\right]\right]d\tau'\,  \Bigg\vert_{\substack{x=1 \\ y=1 \\ z=2 \\ \tau=\infty}}
%    \int_0^{\tau} \frac{2\tau'-2(3+2\tau')\ln(1+\tau')}{(1+2\tau')^3} d\tau'\, = -\frac{3}{4}.
%\end{align}

%Substituting Eq. (\ref{eq:small_rho_num_1}), Eq. (\ref{eq:small_rho_num_2}), and Eq. (\ref{eq:small_rho_num_3}) into Eq. (\ref{eq:lambda_num_contribution}), we obtain
%\begin{align}\label{eq:lambda_num_contribution_solved}
%    \left\langle f_{Ab}f_{aB}f_{AB}\cdot e^{-\frac{f_{A}+f_{B}}{f_0}}\right\rangle \approx&{} -\rho \theta^2 f_0^4 \left[-\frac{1}{2} + \frac{1}{2} + \frac{1}{2} - \frac{3}{4} - \frac{3}{4}\right] = \rho \theta^2 f_0^4.
%\end{align}

To the lowest order in $\theta$, $f_0$, and $\rho$, from Eq. (\ref{eq:lambda_numerator}) the denominator of $\Lambda(f_0)$ follows as 
\begin{align}
    \left\langle f_A^2f_B^2\cdot e^{-\frac{f_{A}+f_{B}}{f_0}}\right\rangle &\approx \theta^2 f_0^4 \partial_x^2 \partial_y^2 H_A H_B \Bigg\vert_{\substack{x=1 \\ y=1 \\ \tau=\infty}} = \theta^2 f_0^4 \frac{1}{x^2y^2}\Bigg\vert_{\substack{x=1 \\ y=1}} = \theta^2 f_0^4.
\end{align}

Thus, in the neutral limit for small $\rho$, $\Lambda(f_0) \approx \rho$.

\section*{Perturbative solution for $\Lambda(f_0)$ for strong selection or recombination}

In the limit that $\gamma_{AB}$ or $\rho$ are large compared to one, we can rescale time in Eq. (\ref{eq:z_general_eq}) so that
\begin{align}\label{eq:z_strong_s}
    \partial_{u} z(u) = -z(u) - \epsilon z^2(u) + \alpha \xi(u)
\end{align}
with the initial condition $z(0) = z$, where $u= \nicefrac{\tau'}{\epsilon}$ is the rescaled time, $\epsilon = \nicefrac{1}{(\gamma_{AB}+\rho)}$, $\alpha = \epsilon \rho$, $\beta_A = \epsilon \gamma_A$, $\beta_B = \epsilon \gamma_B$, and 
\begin{align}
    \xi(u) = \frac{\beta_Axe^{-\beta_A u}}{\beta_A + \epsilon x(1-e^{-\beta_A u})} + \frac{\beta_B ye^{-\beta_B u}}{\beta_B + \epsilon y(1-e^{-\beta_B u})}
\end{align}
is a function independent of $z$.

We can solve Eq. (\ref{eq:z_strong_s}) using a perturbation expansion in $\epsilon$, defining 
\begin{align}
    z (u) &\approx \sum_{i=0}^\infty \epsilon^i z_i (u) \label{eq:z(u)_series}
\end{align}
and, if $\gamma_A,\gamma_B \gg 1$,
\begin{align}
    \xi (u) &\approx \sum_{i=0}^\infty \epsilon^i \xi_i (u) \\ \nonumber
    &= xe^{-\beta_A u} \sum_{i=0}^\infty \left(-\nicefrac{\epsilon}{\beta_A}x(1-e^{-\beta_A u})\right)^i
    + ye^{-\beta_B u} \sum_{i=0}^\infty \left(-\nicefrac{\epsilon}{\beta_B}y(1-e^{-\beta_B u})\right)^i
    .\label{eq:f(u)_series}
\end{align}
In order to find $\Lambda(f_0)$ to the lowers order in $\epsilon$, we need to calculate the first three terms in each series above.
At zeroth order in $\epsilon$,
\begin{align}
    \partial_u z_0(u) = -z_0(u) + \alpha \xi_0(u)  
\end{align}
with the initial condition $z_0(0) = z$ and hence 
% can be solved w/ the method of variation of constants
\begin{align}
    z_0(u) = ze^{-u} + \alpha e^{-u} \int_0^u e^{u'} \xi_0(u') du'.
\end{align}
At first order in $\epsilon$,
\begin{align}
    \partial_u z_1(u) = -z_1(u) -z_0^2(u) + \alpha \xi_1(u) 
\end{align}
with the initial condition $z_1(0) = 0$ and hence 
\begin{align}
    z_1(u) = \alpha e^{-u} \int_0^u e^{u'} \xi_1(u') du' - e^{-u} \int_0^u e^{u'} z_0^2 (u') du'.
\end{align}
At second order in $\epsilon$,
\begin{align}
    \partial_u z_2(u) = -z_2(u) -2z_0(u)z_1(u) + \alpha \xi_2(u) 
\end{align}
with the initial condition $z_2(0) = 0$ and hence 
\begin{align}
    z_2(u) = \alpha e^{-u} \int_0^u e^{u'} \xi_2(u') du' - 2e^{-u} \int_0^u e^{u'} z_0(u')z_1(u') du'.
\end{align}


%If $\gamma_A, \gamma_B \ll 1$, then
%to the second order in $\epsilon$ \begin{align}
%    f(u) \approx \frac{x}{1+\epsilon u x} + \frac{y}{1 + \epsilon u y}
%    % I am not sure about this??
%    \approx x + y - \epsilon u (x^2 + y^2) + \epsilon^2 u^2 (x^3 + y^3).
%\end{align}
%If $\gamma_A, \gamma_B \gg 1$, the first two terms of Eq. (\ref{eq:f(u)_series}) can be found as
% by Taylor expanding the denominators of $\xi(u)$
%\begin{align}
%    \xi_0(u) &= x e^{-\beta_A u} + y e^{-\beta_B u}
%\end{align}
%and
%\begin{align}
%    \xi_1(u) &= -x^2 \frac{1}{\beta_A}e^{-\beta_A u} \left(1-e^{-\beta_Au}\right)
%    -y^2 \frac{1}{\beta_B}e^{-\beta_B u} \left(1-e^{-\beta_Bu}\right).
%\end{align}
%and
%\begin{align}
%    \xi_2(u) &= x^3 \frac{1}{\beta_A^2}e^{-\beta_A u} \left(1-e^{-\beta_Au}\right)^2
%    +y^3 \frac{1}{\beta_B^2}e^{-\beta_B u} \left(1-e^{-\beta_Bu}\right)^2.
%\end{align}
%Hence
%\begin{align}
%    z_0(u) &= z e^{-u} - \frac{\alpha x}{1-\beta_A}\left(e^{-u}-e^{-\beta_A u}\right) - \frac{\alpha y}{1-\beta_B}\left(e^{-u}-e^{-\beta_B u}\right)
%\end{align}
%and
% REWRITE THIS IN POWERS OF Z and introduce g_A, g_B here
%\begin{align}
%    z_1(u) &= 
    %\frac{\alpha x^2}{\beta_A (1-\beta_A)}\left(e^{-u}-e^{-\beta_A u}\right)
    %- \frac{\alpha x^2}{\beta_A (1-2\beta_A)}\left(e^{-u}-e^{-2\beta_A u}\right) \\ \nonumber
    %&+ \frac{\alpha y^2}{\beta_B (1-\beta_B)}\left(e^{-u}-e^{-\beta_B u}\right)
    %- \frac{\alpha y^2}{\beta_B (1-2\beta_B)}\left(e^{-u}-e^{-2\beta_B u}\right) \\ \nonumber
    %&+ \left[z-\frac{\alpha x}{1-\beta_A}-\frac{\alpha y}{1 - \beta_B}\right]^2\left(e^{-2u}-e^{-u}\right) \\ \nonumber
    %&+ \left[z-\frac{\alpha x}{1-\beta_A}-\frac{\alpha y}{1 - \beta_B}\right]\left[\frac{2\alpha x}{\beta_A(1-\beta_A)}\left(e^{-u(1+\beta_A)}-e^{-u}\right) + \frac{2\alpha y}{\beta_B(1-\beta_B)}\left(e^{-u(1+\beta_B)}-e^{-u}\right)\right] \\ \nonumber
    %&- \frac{\alpha^2x^2}{(1-\beta_A)^2(1-2\beta_A)}\left(e^{-2\beta_A u} -e^{-u}\right) - \frac{\alpha^2y^2}{(1-\beta_B)^2(1-2\beta_B)}\left(e^{-2\beta_B u} -e^{-u}\right) \\ \nonumber
    %& -\frac{2\alpha^2 xy}{(1-\beta_A)(1-\beta_B)(1-\beta_A-\beta_B)}\left(e^{-(\beta_A+\beta_B)u}-e^{-u}\right).
%    z^2\left[e^{-2u}-e^{-u}\right] \\ \nonumber
%    &- \frac{2 \alpha xz}{1-\beta_A}\left[e^{-2u} - \left(1-\frac{1}{\beta_A}\right)e^{-u}-\frac{1}{\beta_A}e^{-u(1+\beta_A)}\right] \\ \nonumber
%    &- \frac{2 \alpha yz}{1-\beta_B}\left[e^{-2u} - \left(1-\frac{1}{\beta_B}\right)e^{-u}-\frac{1}{\beta_B}e^{-u(1+\beta_B)}\right] \\ \nonumber
%    &+ \frac{2\alpha^2 xy}{(1-\beta_A)(1-\beta_B)}\left[e^{-2u} - \left(1-\frac{1}{\beta_A} -\frac{1}{\beta_B} - \frac{1}{1-\beta_A-\beta_B}\right)e^{-u} \right. \\ \nonumber &-\left.\frac{1}{\beta_A}e^{-u(1+\beta_A)}-\frac{1}{\beta_B}e^{-u(1+\beta_B)}
%    -\frac{1}{1-\beta_A-\beta_B}e^{-u(\beta_A+\beta_B)}\right] \\ \nonumber
%    & + \alpha x^2 \left[\frac{1}{1-2\beta_A}\left(\frac{2\alpha}{\beta_A}-\frac{1}{1-\beta_A}\right)e^{-u} +\frac{\alpha}{(1-\beta_A)^2}e^{-2u} - \frac{2\alpha}{\beta_A(1-\beta_A)^2}e^{-u(1-\beta_A)} \right. \\ \nonumber
%    &- \left. \frac{1}{\beta_A(1-\beta_A)}e^{-\beta_A u} - \frac{1}{1-2\beta_A}\left(\frac{1}{\beta_A}+\frac{1}{(1-\beta_A)^2}\right)e^{-2\beta_A u}\right] \\ \nonumber
%    & + \alpha y^2 \left[\frac{1}{1-2\beta_B}\left(\frac{2\alpha}{\beta_B}-\frac{1}{1-\beta_B}\right)e^{-u} +\frac{\alpha}{(1-\beta_B)^2}e^{-2u} - \frac{2\alpha}{\beta_B(1-\beta_B)^2}e^{-u(1-\beta_B)} \right. \\ \nonumber
%    &- \left. \frac{1}{\beta_B(1-\beta_B)}e^{-\beta_B u} - \frac{1}{1-2\beta_B}\left(\frac{1}{\beta_B}+\frac{1}{(1-\beta_B)^2}\right)e^{-2\beta_B u}\right].
%\end{align}

%To the lowest order in $\epsilon$, the the numerator of $\Lambda(f_0)$ then follows as
%\begin{align}\label{eq:lambda_num_large_rho}
%    \left\langle f_{Ab}f_{aB}f_{AB}\cdot e^{-\frac{f_{A}+f_{B}}{f_0}}\right\rangle &\approx -\epsilon \theta^2 f_0^4
%    \int_0^{\nicefrac{\tau}{\epsilon}} du\, \partial_x \partial_y \partial_z (z_0 + \epsilon z_1) \left(\Phi_x + \Phi_y - \nicefrac{\alpha}{\epsilon}\Phi_x\Phi_y\right)\Bigg\vert_{\substack{x=1 \\ y=1 \\ z=2 \\ \tau=\infty}} \\\nonumber
%    &=\alpha \theta^2 f_0^4\left[
%    \int_0^{\nicefrac{\tau}{\epsilon}} e^{-u} du \, \partial_x \Phi_x \, \partial_y \Phi_y \, \partial_z z \right. \\\nonumber
%    &+\left.\frac{2\epsilon^2}{1-\beta_A}
%    \int_0^{\nicefrac{\tau}{\epsilon}} \xi_{\beta_A} du\, \partial_y \Phi_y \, \partial_x \partial_z xz \right. \\\nonumber
%    &+\left.\frac{2\epsilon^2}{1-\beta_B}
%    \int_0^{\nicefrac{\tau}{\epsilon}} \xi_{\beta_B} du\, \partial_x \Phi_x \, \partial_y \partial_z yz \right. \\\nonumber
%    &- \left.\epsilon 
%    \int_0^{\nicefrac{\tau}{\epsilon}} (e^{-2u} - e^{-u}) du\, \partial_x \Phi_x \partial_y \Phi_y \partial_z z^2 \right. \\\nonumber
%    &- \left.\frac{2\epsilon\alpha}{1-\beta_A} 
%    \int_0^{\nicefrac{\tau}{\epsilon}} \xi_{\beta_A} du\, \partial_y \Phi_y \partial_x \partial_z z\Phi_x\right. \\\nonumber
%    &- \left.\frac{2\epsilon\alpha}{1-\beta_A} 
%    \int_0^{\nicefrac{\tau}{\epsilon}} \xi_{\beta_B} du\, \partial_x \Phi_x \partial_y \partial_z z\Phi_y\right] \Bigg\vert_{\substack{x=1 \\ y=1 \\ z=2 \\ \tau=\infty}},
%\end{align}

To the lowest order in $\epsilon$, the numerator of $\Lambda(f_0)$ follows from Eq. (\ref{eq:lambda_numerator}) as
\begin{align}\label{eq:lambda_num_large_rho}
    \left\langle f_{Ab}f_{aB}f_{AB}\cdot e^{-\frac{f_{A}+f_{B}}{f_0}}\right\rangle
 &\approx -\theta^2 f_0^4 \int_0^{\nicefrac{\tau}{\epsilon}} du\, \partial_x\partial_y\partial_z \left[-\alpha z_0\Phi_x\Phi_y -\alpha \epsilon z_1 \Phi_x\Phi_y -\alpha \epsilon^2 z_2 \Phi_x\Phi_y \right. \\ \nonumber
 &+ \left. \epsilon^2 z_1\Phi_x + \epsilon^2 z_1\Phi_y + \epsilon^3 z_2\Phi_x + \epsilon^3 z_2\Phi_y \right]\Bigg\vert_{\substack{x=1 \\ y=1 \\ z=2 \\ \tau=\infty}} \\ \nonumber
 %&\approx \alpha \epsilon^4 \theta^2 f_0^4 \int_0^{\nicefrac{\tau}{\epsilon}} du\,\Bigg[ e^{-u}\partial_x \Phi_x \partial_y \Phi_y \partial_z z 
 %+ \epsilon(e^{-2u} - e^{-u})\partial_x \Phi_x \partial_y \Phi_y \partial_z z^2 
 %\Bigg. \\ \nonumber
 %& - \epsilon \frac{2\alpha}{1-\beta_A} g_A \partial_x x\Phi_x \partial_y \Phi_y \partial_z z - \epsilon \frac{2\alpha}{1-\beta_B} g_B \partial \Phi_x \partial_y y \Phi_y \partial_z z\Bigg. \\ \nonumber
 %&+ \Bigg. \epsilon^2 \frac{2}{1-\beta_B}g_B \partial_x \Phi_x \partial_y y \partial_z z 
 %+ \epsilon^2 \frac{2}{1-\beta_A}g_A \partial_x x \partial_y \Phi_y \partial_z z \Bigg]\Bigg\vert_{\substack{x=1 \\ y=1 \\ z=2 \\ \tau=\infty}} \\ \nonumber
 &\approx \frac{\alpha \epsilon^4 \theta^2 f_0^4}{\beta_A^2 \beta_B^2 (1+\beta_A+\beta_B)} \left[1 + \frac{\beta_A(\alpha+\beta_A)}{(1+\beta_B)(1+\nicefrac{\beta_B}{2}) } +  \frac{\beta_B(\alpha+\beta_B)}{(1+\beta_A)(1+\nicefrac{\beta_A}{2})} \right. \\ \nonumber
 &+\left. \frac{2 \alpha \beta_A \beta_B (\alpha+\beta_A+\beta_B)(2+\beta_A+\beta_B)}{(1+\beta_A)(1+\beta_B)} \right]+ \mathcal{O}(\epsilon^5),
%    &= \frac{\epsilon^2\alpha\theta^2f_0^4}{\beta_A^2\beta_B^2(1+\beta_A+\beta_B)},
\end{align}
%where we have defined
%\begin{subequations}\begin{align}
%    \xi_{\beta_A} &\equiv e^{-2u} - \left(1-\frac{1}{\beta_A}\right)e^{-u}-\frac{1}{\beta_A}e^{-u(1+\beta_A)}, \\
%    \xi_{\beta_B} &\equiv e^{-2u} - \left(1-\frac{1}{\beta_B}\right)e^{-u}-\frac{1}{\beta_B}e^{-u(1+\beta_B)}.    
%\end{align}\end{subequations}
%where we have defined
%\begin{subequations}\begin{align}
%    g_A(u) & = e^{-2u} - \left(1-\frac{1}{\beta_A}\right)e^{-u}-\frac{1}{\beta_A}e^{-u(1+\beta_A)}, \\
%    g_B(u) & = e^{-2u} - \left(1-\frac{1}{\beta_B}\right)e^{-u}-\frac{1}{\beta_B}e^{-u(1+\beta_B)}
%\end{align}\end{subequations}
where we have used used
\begin{subequations}\begin{align}
    \Phi_x \Big\vert_{\substack{\tau=\infty}} &\approx - \frac{\epsilon}{\beta_A} + \mathcal{O}(\epsilon^2)
    %\frac{\epsilon^2}{\beta_A^2}xe^{-\beta_Au} +\mathcal{O}(\epsilon^3)
    , \\
    \Phi_y \Big\vert_{\substack{\tau=\infty}} &\approx - \frac{\epsilon}{\beta_B} + \mathcal{O}(\epsilon^2)
    %\frac{\epsilon^2}{\beta_B^2}ye^{-\beta_Bu} +\mathcal{O}(\epsilon^3)
    , \\
    \partial_x \Phi_x \Big\vert_{\substack{\tau=\infty}} &\approx \frac{\epsilon^2}{\beta_A^2}e^{-\beta_Au} +\mathcal{O}(\epsilon^3), \\
    \partial_y \Phi_y \Big\vert_{\substack{\tau=\infty}} &\approx \frac{\epsilon^2}{\beta_B^2}e^{-\beta_Bu} +\mathcal{O}(\epsilon^3).
\end{align}\end{subequations}
%\begin{subequations}\begin{align}
%    \partial_x \Phi_x \Bigg\vert_{\substack{x=1 \\ \tau=\infty}} &= e^{-\beta_A u} \left[\frac{1-e^{-\beta_A(\nicefrac{\tau}{\epsilon}-u)}}{x(1-e^{-\beta_A \nicefrac{\tau}{\epsilon}}) + \nicefrac{\beta_A}{\epsilon}}\right]^2\Bigg\vert_{\substack{x=1 \\ \tau=\infty}} = \frac{\epsilon^2}{\beta_A^2}e^{-\beta_A u}, \\
%    \partial_y \Phi_y \Bigg\vert_{\substack{y=1 \\ \tau=\infty}} &= e^{-\beta_B u} \left[\frac{1-e^{-\beta_B(\nicefrac{\tau}{\epsilon}-u)}}{y(1-e^{-\beta_B \nicefrac{\tau}{\epsilon}}) + \nicefrac{\beta_B}{\epsilon}}\right]^2\Bigg\vert_{\substack{y=1 \\ \tau=\infty}} = \frac{\epsilon^2}{\beta_B^2}e^{-\beta_B u},\\
%    \Phi_x \Bigg\vert_{\substack{x=1 \\ \tau=\infty}} &= -\frac{\nicefrac{\beta_A}{\epsilon}+(1-e^{-\beta_A u})}{\left(\nicefrac{\beta_A}{\epsilon}\right)^2}, \\
%    \Phi_y \Bigg\vert_{\substack{y=1 \\ \tau=\infty}} &= -\frac{\nicefrac{\beta_B}{\epsilon}+(1-e^{-\beta_B u})}{\left(\nicefrac{\beta_B}{\epsilon}\right)^2}.
%\end{align}\end{subequations}

The denominator of $\Lambda(f_0)$ follows from Eq. (\ref{eq:lambda_numerator}) as 
\begin{align}
    \left\langle f_A^2f_B^2\cdot e^{-\frac{f_{A}+f_{B}}{f_0}}\right\rangle &\approx \theta^2 f_0^4 \partial_x^2 \partial_y^2 H_A H_B \Bigg\vert_{\substack{x=1 \\ y=1 \\ \tau=\infty}} = \frac{\epsilon^4 \theta^2 f_0^4}{\beta_A^2 \beta_B^2} \frac{1}{x^2y^2}\Bigg\vert_{\substack{x=1 \\ y=1}} = \frac{\epsilon^4 \theta^2 f_0^4}{\beta_A^2 \beta_B^2}.
\end{align}

$\Lambda(f_0)$ then follows as
\begin{align}\label{eq:lambda_strong_s}
    \Lambda(f_0) &\approx \frac{\alpha}{1+\beta_A + \beta_B} \\ \nonumber
    &\times \left[1 +  \frac{2\beta_A(\alpha+\beta_A)}{(1+\beta_B)(2+\beta_B)}
    +\frac{2\beta_B(\alpha+\beta_B)}{(1+\beta_A)(2+\beta_A)} + \frac{2 \alpha \beta_A \beta_B (\alpha+\beta_A+\beta_B)(2+\beta_A+\beta_B)}{(1+\beta_A)(1+\beta_B)}\right] \\ \nonumber
    &= \frac{\rho}{\rho + \gamma_A + \gamma_B + \gamma_{AB}} \\\nonumber
    &\times \left[1 +\frac{\gamma_A(\rho + \gamma_{A})}{(\rho + \gamma_B + \gamma_{AB})(\rho + \nicefrac{1}{2}\gamma_B + \gamma_{AB})} 
    +\frac{\gamma_B(\rho + \gamma_{B})}{(\rho + \gamma_A + \gamma_{AB})(\rho + \nicefrac{1}{2}\gamma_A + \gamma_{AB})} \right. \\ \nonumber
    &+\left. \frac{4 \rho \gamma_A \gamma_B (\rho+\gamma_A+\gamma_B)(\rho+\nicefrac{1}{2}\gamma_A+\nicefrac{1}{2}\gamma_B+\gamma_{AB})}{(\rho+\gamma_A+\gamma_{AB})(\rho+\gamma_B+\gamma_{AB})(\rho+\gamma_{AB})^3}\right].
%\Lambda(f_0) &\approx \frac{\alpha}{1+\beta_A + \beta_B} = \frac{\rho}{\rho + \gamma_A + \gamma_B + \gamma_{AB}}.
\end{align}

The approximation above holds for any $\rho$ as long as $\gamma_{A}, \gamma_{B} \gg 1$. If $\rho \gg \gamma_A, \gamma_B$, then 
\begin{align}
    \Lambda(f_0) &\approx 1.
\end{align}
If $\rho \ll \gamma_A, \gamma_B$, $\gamma_A=\gamma_B=\gamma$, $\gamma_{AB}=2\gamma$, then
\begin{align}
    \Lambda(f_0) &\approx \frac{19}{60}\frac{\rho}{\gamma}.
\end{align} 

In the limit that $\gamma_A \gg 1,\gamma_B =0$, we can solve Eq. (\ref{eq:z_strong_s}) expanding
\begin{align}
    \xi (u) &\approx \sum_{i=0}^\infty \epsilon^i \xi_i (u) = xe^{-\beta_A u} \sum_{i=0}^\infty \left(-\nicefrac{\epsilon}{\beta_A}x(1-e^{-\beta_A u})\right)^i
    + y \sum_{i=0}^\infty \left(-\epsilon u\right)^i
    .\label{eq:f(u)_series}
\end{align}

To the lowest order in $\epsilon$, the numerator of $\Lambda(f_0)$ follows from Eq. (\ref{eq:lambda_numerator}) as
\begin{align}\label{eq:lambda_num_large_rho}
    \left\langle f_{Ab}f_{aB}f_{AB}\cdot e^{-\frac{f_{A}+f_{B}}{f_0}}\right\rangle
 &\approx -\theta^2 f_0^4 \int_0^{\nicefrac{\tau}{\epsilon}} du\, \partial_x\partial_y\partial_z \left[-\alpha z_0\Phi_x\Phi_y -\alpha \epsilon z_1 \Phi_x\Phi_y + \epsilon^2 z_1\Phi_y \right]\Bigg\vert_{\substack{x=1 \\ y=1 \\ z=2 \\ \tau=\infty}} \\ \nonumber
 &\approx \frac{\alpha \epsilon^2 \theta^2 f_0^4}{\beta_A^2 (1+\beta_A)} \left[1 - \beta_A(\alpha+\beta_A) \right]+ \mathcal{O}(\epsilon^3),
\end{align}
where we have used used
\begin{subequations}\begin{align}
    \Phi_x \Big\vert_{\substack{\tau=\infty}} &\approx - \frac{\epsilon}{\beta_A} + \mathcal{O}(\epsilon^2)
    %\frac{\epsilon^2}{\beta_A^2}xe^{-\beta_Au} +\mathcal{O}(\epsilon^3)
    , \\
    \Phi_y \Big\vert_{\substack{\tau=\infty}} &\approx - \frac{1}{y} + \mathcal{O}(\epsilon)
    %\frac{\epsilon^2}{\beta_B^2}ye^{-\beta_Bu} +\mathcal{O}(\epsilon^3)
    , \\
    \partial_x \Phi_x \Big\vert_{\substack{\tau=\infty}} &\approx \frac{\epsilon^2}{\beta_A^2}e^{-\beta_Au} +\mathcal{O}(\epsilon^3), \\
    \partial_y \Phi_y \Big\vert_{\substack{\tau=\infty}} &\approx \frac{1}{y^2} +\mathcal{O}(\epsilon).
\end{align}\end{subequations}

The denominator of $\Lambda(f_0)$ follows from Eq. (\ref{eq:lambda_numerator}) as 
\begin{align}
    \left\langle f_A^2f_B^2\cdot e^{-\frac{f_{A}+f_{B}}{f_0}}\right\rangle &\approx \theta^2 f_0^4 \partial_x^2 \partial_y^2 H_A H_B \Bigg\vert_{\substack{x=1 \\ y=1 \\ \tau=\infty}} = \frac{\epsilon^2 \theta^2 f_0^4}{\beta_A^2} \frac{1}{x^2y^2}\Bigg\vert_{\substack{x=1 \\ y=1}} = \frac{\epsilon^2 \theta^2 f_0^4}{\beta_A^2}.
\end{align}

$\Lambda(f_0)$ then follows as
\begin{align}\label{eq:lambda_strong_s_neutral}
    \Lambda(f_0) &\approx \frac{\alpha}{1+\beta_A} 
     \left[1 -  \beta_A(\alpha+\beta_A)\right] = \frac{\rho}{\rho + \gamma_A + \gamma_{AB}} \left[1 -\frac{\rho\gamma_A(\rho + \gamma_{A})}{(\rho+\gamma_{AB})^3}\right].
%\Lambda(f_0) &\approx \frac{\alpha}{1+\beta_A + \beta_B} = \frac{\rho}{\rho + \gamma_A + \gamma_B + \gamma_{AB}}.
\end{align}

If $\rho \gg \gamma_A$, then 
\begin{align}
    \Lambda(f_0) &\approx 1.
\end{align}
If $\rho \ll \gamma_A$, $\gamma_{AB}=\gamma_A = \gamma$, then
\begin{align}
    \Lambda(f_0) &\approx \frac{1}{2}\frac{\rho}{\gamma}.
\end{align} 


% It seems that the only two regimes that we have not thought about yet are $\rho \gg 1$, $\gamma_A, \gamma_B \ll 1$ and $\rho \ll s \ll 1$ (not sure if this is interesting?).
%If $\rho \gg 1$, but all mutations are neutral, 
%\begin{align}\label{eq:z_strong_r}
%    \partial_{u} z(u) = -z(u) - \epsilon z^2(u) + \xi(u)
%\end{align}
%with the initial condition $z(0) = z$, where $u= \nicefrac{\tau'}{\epsilon}$, $\epsilon = \nicefrac{1}{\rho}$, and
%\begin{align}
%    \xi(u) = \frac{x}{1+\epsilon u x} + \frac{y}{1+\epsilon u y} \approx x + y - \epsilon u (x^2 + y^2) + \mathcal{O}(\epsilon^3).
%\end{align}
\end{document}
