\documentclass[11pt]{article}

\usepackage{amsmath,amsfonts,amssymb,mathtools,nicefrac,cases,empheq,enumitem}
\usepackage{xcolor,tikz,graphicx}

% Formatting
\usepackage[utf8]{inputenc}
\usepackage[top=1in, bottom=1in, right=1in, left=1in]{geometry} % page margins
\setlength{\parindent}{0 pt} % paragraph left indentation
\setlength{\parskip}{36 pt} % before paragraph spacing
\setlength{\jot}{12pt} % between-line spacing in multi-line equations
\renewcommand{\baselinestretch}{1.25} % between-line spacing
\usepackage{titlesec,float} % sections spacing + indent first paragraph in section, placing figures
\titlespacing*{\section}{0 pt}{0 pt}{0 pt} % title spacing
\titlespacing*{\subsection}{0 pt}{0 pt}{0 pt}
\setlength{\abovecaptionskip}{6 pt} % moving figure captures up
%\setlength{\abovecaptionskip}{-12 pt} % reducing space below figures
\allowdisplaybreaks


\title{\vspace{-36 pt} \Large Perturbative solution of the $z(\tau')$ characteristic in the limit that $\rho$ is small \vspace{-36 pt}}
\date{}

\begin{document}
%\maketitle
We are interested in the linkage equilibrium statistic $\Lambda(f_0)$, defined as
\begin{align}
    \Lambda(f_0) \equiv \frac{\left\langle f_{ab}f_{Ab}f_{aB}f_{AB} \cdot e^{-\frac{f_{A}+f_{B}}{f_0}}\right\rangle}{\left\langle f_A^2(1-f_A)^2f_B^2(1-f_B)^2\cdot e^{-\frac{f_{A}+f_{B}}{f_0}}\right\rangle},
\end{align}
where $f_A \equiv f_{Ab} + f_{aB}$ and $f_B \equiv f_{aB} + f_{AB}$.
In the limit that $f_{Ab}$, $f_{aB}$, $f_{AB} \ll1$, 
\begin{align}\label{eq:lambda_small_f}
    \Lambda(f_0) \approx \frac{\left\langle f_{Ab}f_{aB}f_{AB} \cdot e^{-\frac{f_{A}+f_{B}}{f_0}}\right\rangle}{\left\langle f_A^2f_B^2\cdot e^{-\frac{f_{A}+f_{B}}{f_0}}\right\rangle}.
\end{align}
The moments follows from 
\begin{align}\label{eq:lambda_numerator}
    \left\langle f_{Ab}^if_{aB}^jf_{AB}^k\cdot e^{-\frac{f_{A}+f_{B}}{f_0}}\right\rangle
    = -f_0^{(i+j+k)} \partial_x^i \partial_y^j \partial_z^k H(x, y, z, t) \Bigg\vert_{\substack{x=1 \\ y=1 \\ z=2 \\ t=\infty}},
\end{align}
where %$m = i+j+k$ and 
\begin{align}
    H(x, y, z, t) \equiv \left\langle e^{-x\frac{f_{Ab}(t)}{f_0}-y\frac{f_{aB}(t)}{f_0}-z\frac{f_{AB}(t)}{f_0}} \right\rangle 
\end{align}
is the joint moment generating function. 
%The weighted moment in the numerator of Eq.(\ref{eq:lambda_small_f}) can be obtained as
%\begin{align}\label{eq:lambda_numerator}
%    \left\langle f_{Ab}f_{aB}f_{AB} \cdot e^{-\frac{f_{A}+f_{B}}{f_0}}\right\rangle
%    = -f_0^3 \frac{\partial^3}{\partial x \partial y \partial z} \lim_{t\to\infty} H(x+1, y+1, z+2, t).
%\end{align}
%and 
%\begin{align}\label{eq:lambda_denominator}
%    \left\langle f_{A}f_{B} \cdot e^{-\frac{f_{A}+f_{B}}{f_0}}\right\rangle &=
    %\left\langle f_{Ab}f_{aB} \cdot e^{-\frac{f_{A}+f_{B}}{f_0}}\right\rangle 
    %+ \left\langle f_{Ab}f_{AB} \cdot e^{-\frac{f_{A}+f_{B}}{f_0}}\right\rangle \\
    %&\quad + \left\langle f_{aB}f_{AB} \cdot e^{-\frac{f_{A}+f_{B}}{f_0}}\right\rangle 
    %+ \left\langle f_{AB}^2 \cdot e^{-\frac{f_{A}+f_{B}}{f_0}}\right\rangle \\
    %& \quad = 
%    f_0^2 \left[\frac{\partial^2}{\partial x \partial y} + \frac{\partial^2}{\partial x \partial z} + \frac{\partial^2}{\partial y \partial z} + \frac{\partial^2}{\partial z^2}\right] \lim_{t\to\infty} H(x+1, y+1, z+2, t).
%\end{align}


In the limit that $\theta = 2N\mu$ and $f_0$ are both small compared to one, 
\begin{align}\label{eq:h_expansion}
    H(x, y, z, \tau) &\approx 1 -\theta (H_A + H_B)
    + \frac{\theta^2}{2}\left(H_A + H_B\right)^2 \\\nonumber
    &\quad + \theta^2f_0\Upsilon + \theta^2f_0\int_0^{\tau}d \tau' z(\tau') \left[\Phi_x(\tau')+\Phi_y(\tau')-\rho\Phi_x(\tau')\Phi_y(\tau')\right] \\\nonumber
    &\quad + \mathcal{O}(f_0^2) +\mathcal{O}(\theta^3),
\end{align}
where $\tau = \nicefrac{t}{2Nf_0}$, $\gamma_A = 2Ns_Af_0$, $\gamma_B = 2Ns_Bf_0$, $\rho = 2NRf_0$, and 
\begin{subequations}\begin{align}
    H_A(x, \tau) &\equiv \ln \left[1 + \frac{x(1-e^{-\gamma_A\tau})}{\gamma_A}\right], \\
    H_B(y, \tau) &\equiv \ln \left[1 + \frac{y(1-e^{-\gamma_B\tau})}{\gamma_B}\right],
\end{align}\end{subequations}
\begin{subequations}\begin{align}
    \Phi_x(\tau') &\equiv -\frac{[ 1-e^{-\gamma_A (\tau-\tau')} ][\gamma_A+x(1-e^{-\gamma_A \tau'})]}{\gamma_A \left[ \gamma_A+x(1-e^{-\gamma_A \tau}) \right]}, \\
    \Phi_y(\tau') &\equiv -\frac{[1-e^{-\gamma_B (\tau-\tau')}][\gamma_B+y(1-e^{-\gamma_B \tau'})]}{\gamma_A \left[ \gamma_B+y(1-e^{-\gamma_B \tau}) \right]},
\end{align}\end{subequations}
\begin{align}
    \Upsilon(x, y, \tau) &= \int_0^{\tau} d\tau' \rho \left[x(\tau') + y(\tau')\right]\Phi_x(\tau')\Phi_y(\tau').
\end{align}

The characteristic $z(\tau')$ is defined by 
\begin{align}\label{eq:z_general_eq}
    \partial_{\tau'} z(\tau') = -(\gamma_{AB} + \rho) z(\tau') - z^2(\tau') + \rho \frac{\gamma_Axe^{-\gamma_A\tau'}}{\gamma_A+x(1-e^{-\gamma_A\tau'})} + \rho \frac{\gamma_Bye^{-\gamma_B\tau'}}{\gamma_B+y(1-e^{-\gamma_B\tau'})}
\end{align}
with the initial condition $z(0)=z$, where $\gamma_{AB} = \gamma_A + \gamma_B + \gamma_{\epsilon}$. In the absence of recombination, the equation above has an exact solution, 
\begin{align}
    z_0(\tau') = \frac{\gamma_{AB}ze^{-\gamma_{AB}\tau'}}{\gamma_{AB}+z(1-e^{-\gamma_{AB}\tau'})}.
\end{align}
In the limit that $\rho \ll 1$, corrections to the zeroth-order solution can be found by perturbatively expanding $z(\tau')$ as
\begin{align}\label{eq:z_series}
    z(\tau') \approx z_0 + \sum_{i=1}^{\infty} \rho^{i}z_i.
\end{align}
Plugging the series expansion in the equation for $z(\tau')$ and matching the coefficients in front powers of $\rho$, we obtain 
\begin{align}
        \partial_{\tau'}z_1(\tau') \approx -\gamma_{AB}z_1(\tau') -2z_0(\tau')z_1(\tau') - z_0(\tau') + \frac{\gamma_Axe^{-\gamma_A\tau'}}{\gamma_A+x(1-e^{-\gamma_A\tau'})} + \frac{\gamma_Bye^{-\gamma_B\tau'}}{\gamma_B+y(1-e^{-\gamma_B\tau'})}.
\end{align}

%\section*{Perturbative solution of $\Lambda(f_0)$ in the neutral limit for small $\rho$}
In the neutral limit, the equation above reduces to
\begin{align}\label{eq:z_neutral_eq}
    \partial_{\tau'}z_1(\tau') &\approx - \frac{2z}{1+z\tau'}z_1(\tau') - \frac{z}{1+z\tau'} + \frac{x}{1+x\tau'} + \frac{y}{1+y\tau'}.
\end{align}
This inhomogeneous linear ordinary differential equation can be solved by the method of variation of constants. The corresponding homogeneous equation 
\begin{align}
    \partial_{\tau'}z_1(\tau') &\approx - \frac{2z}{1+z\tau'}z_1(\tau')
\end{align}
has solution in the form
\begin{align}\label{eq:z_1_homogeneous_u}
    z_1(\tau') \approx \frac{u(\tau')}{(1+z\tau')^2},
\end{align}
where $u(\tau')$ is some function of $\tau'$. Plugging Eq. (\ref{eq:z_1_homogeneous_u}) into Eq. (\ref{eq:z_neutral_eq}), we obtain
\begin{align}
    \partial_{\tau'} u(\tau') \approx - z(1+z\tau') + \frac{x(1+z\tau')^2}{1+x\tau'} + \frac{y(1+z\tau')^2}{1+y\tau'},
\end{align}
from where
\begin{align}
    u(\tau') &\approx - \int z(1+z\tau') \,d\tau'\
    + \int \frac{x(1+z\tau')^2}{1+x\tau'} \,d\tau'\
    + \int \frac{y(1+z\tau')^2}{1+y\tau'} \,d\tau'\, \\\nonumber
    &= \frac{1}{2} (1+z\tau')^2 + z\tau'\left(1 - \frac{z}{x}\right) + z\tau'\left(1 - \frac{z}{y}\right)
     \\\nonumber
    &\quad + \left(1-\frac{z}{x}\right)^2\ln(1+x\tau')
    + \left(1-\frac{z}{y}\right)^2\ln(1+y\tau') + C,
\end{align}
where 
\begin{align}
    C = \frac{1}{2}
\end{align} is a constant determined by the initial condition $u(0)=0$. Then, 
\begin{align}\label{eq:z_1}
    z_1(\tau') &\approx \frac{1}{2} 
    + \frac{1}{2}\frac{1}{(1+z\tau')^2} + \left(1-\frac{z}{x}\right)\frac{z\tau'}{(1+z\tau')^2} + \left(1-\frac{z}{y}\right)\frac{z\tau'}{(1+z\tau')^2}
         \\\nonumber
    &\quad + \left(1-\frac{z}{x}\right)^2 \frac{\ln(1+x\tau')}{(1+z\tau')^2}
    + \left(1-\frac{z}{y}\right)^2 \frac{\ln(1+y\tau')}{(1+z\tau')^2}.
\end{align}
Substituting Eq. (\ref{eq:z_1}) into Eq. (\ref{eq:z_series}), we find
\begin{align}
    z(\tau') &\approx \frac{z}{1+z\tau'} 
    + \frac{\rho}{2}
    + \frac{\rho}{2}\frac{1}{(1+z\tau')^2} 
    \\\nonumber
    &\quad + \rho\left(1-\frac{z}{x}\right)\frac{z\tau'}{(1+z\tau')^2} + \rho\left(1-\frac{z}{y}\right)\frac{z\tau'}{(1+z\tau')^2}
    \\\nonumber
    &\quad + \rho\left(1-\frac{z}{x}\right)^2 \frac{\ln(1+x\tau')}{(1+z\tau')^2}
    + \rho\left(1-\frac{z}{y}\right)^2 \frac{\ln(1+y\tau')}{(1+z\tau')^2}.
\end{align}

Thus, to the lowest order in $\rho$, the numerator of $\Lambda(f_0)$ follows as
\begin{align}\label{eq:lambda_num_contribution}
    \left\langle f_{Ab}f_{aB}f_{AB}\cdot e^{-\frac{f_{A}+f_{B}}{f_0}}\right\rangle \approx&{} -\rho \theta^2 f_0^4 \left[
    -\int_0^{\tau} d\tau'\, \partial_x \Phi_x \, \partial_y \Phi_y \, \partial_z \frac{z}{1+z\tau'} \right. \\\nonumber
    &- \left. \int_0^{\tau} d\tau'\, \partial_x \Phi_x \, \partial_y \partial_z \frac{z}{y} \frac{z\tau'}{(1+z\tau')^2} \right. \\\nonumber
    &- \left. \int_0^{\tau} d\tau'\, \partial_y \Phi_y \, \partial_x \partial_x \frac{z}{x} \frac{z\tau'}{(1+z\tau')^2} \right. \\\nonumber
    &+ \left. \int_0^{\tau} d\tau'\, \partial_x \Phi_x \, \partial_y \partial_z \left(1-\frac{z}{y}\right)^2 \frac{\ln(1+y\tau')}{(1+z\tau')^2} \right. \\\nonumber
    &+ \left. \int_0^{\tau} d\tau'\, \partial_y \Phi_y \, \partial_x \partial_z \left(1-\frac{z}{x}\right)^2 \frac{\ln(1+x\tau')}{(1+z\tau')^2} 
    \right]\Bigg\vert_{\substack{x=1 \\ y=1 \\ z=2 \\ \tau=\infty}},
\end{align}
where 
\begin{subequations}\begin{align}
    %\Phi_x(\tau') &= -\frac{(\tau-\tau')(1+x\tau')}{1+x\tau}
    \partial_x \Phi_x \Bigg\vert_{\substack{x=1 \\ \tau=\infty}} = 
    %-(\tau-\tau') \partial_x \frac{1+x\tau'}{1+x\tau} \Bigg\vert_{\substack{x=1 \\ \tau=\infty}} =
    -\frac{(\tau-\tau')^2}{(1+x\tau)^2} \Bigg\vert_{\substack{x=1 \\ \tau=\infty}} = 1, \\
    %\Phi_y(\tau') &= -\frac{(\tau-\tau')(1+y\tau')}{1+y\tau}.
    \partial_y \Phi_y \Bigg\vert_{\substack{y=1 \\ \tau=\infty}} =
    %-(\tau-\tau') \partial_x \frac{1+x\tau'}{1+x\tau} \Bigg\vert_{\substack{x=1 \\ \tau=\infty}} =
    -\frac{(\tau-\tau')^2}{(1+y\tau)^2} \Bigg\vert_{\substack{y=1 \\ \tau=\infty}} = 1.
\end{align}\end{subequations}

The first integral in Eq. (\ref{eq:lambda_num_contribution}) evaluates to
\begin{align}\label{eq:small_rho_num_1}
    -\int_0^{\tau} d\tau'\, \partial_x \Phi_x \, \partial_y \Phi_y \, \partial_z \frac{z}{1+z\tau'}\Bigg\vert_{\substack{x=1 \\ y=1 \\ z=2 \\ \tau=\infty}} & = -\int_0^{\infty} \frac{1}{(1+2\tau')^2}\,d\tau = -\frac{1}{2}.
\end{align}

The second (and third) integral in Eq. (\ref{eq:lambda_num_contribution}) can be evaluated as \begin{align}\label{eq:small_rho_num_2}
    & -\int_0^{\tau} d\tau'\, \partial_x \Phi_x \, \partial_y \partial_z \frac{z}{y} \frac{z\tau'}{(1+z\tau')^2} \Bigg\vert_{\substack{x=1 \\ y=1 \\ z=2 \\ \tau=\infty}} 
    %= -\int_0^{\tau} d\tau'\, \partial_y \Phi_y \, \partial_x \partial_z \frac{z}{x} \frac{z\tau'}{(1+z\tau')^2} \Bigg\vert_{\substack{x=1 \\ y=1 \\ z=2 \\ \tau=\infty}} 
    %= \int_0^{\infty} \frac{2z\tau'}{(1+z\tau')^3} \,d\tau'
    = \int_0^{\infty} \frac{4\tau'}{(1+2\tau')^3} \,d\tau' = \frac{1}{2}.
\end{align}

Finally, we find the last (and the second to last) integral in Eq. (\ref{eq:lambda_num_contribution}) as
\begin{align}\label{eq:small_rho_num_3}
    \int_0^{\tau} d\tau'\, \partial_y \Phi_y \, \partial_x \partial_z \left(1-\frac{z}{x}\right)^2 \frac{\ln(1+x\tau')}{(1+z\tau')^2} \Bigg\vert_{\substack{x=1 \\ y=1 \\ z=2 \\ \tau=\infty}} &= %\int_0^{\tau} \left[-\frac{2\tau'(x-z)}{x^2(1+z\tau')^3}-\frac{2\ln(1+x\tau')}{x^3(1+z\tau')^3}\left[z(1+x\tau')-(x-z)\right]\right]d\tau'\,  \Bigg\vert_{\substack{x=1 \\ y=1 \\ z=2 \\ \tau=\infty}}
    \int_0^{\tau} \frac{2\tau'-2(3+2\tau')\ln(1+\tau')}{(1+2\tau')^3} d\tau'\, = -\frac{3}{4}.
\end{align}

Substituting Eq. (\ref{eq:small_rho_num_1}), Eq. (\ref{eq:small_rho_num_2}), and Eq. (\ref{eq:small_rho_num_3}) into Eq. (\ref{eq:lambda_num_contribution}), we obtain
\begin{align}\label{eq:lambda_num_contribution_solved}
    \left\langle f_{Ab}f_{aB}f_{AB}\cdot e^{-\frac{f_{A}+f_{B}}{f_0}}\right\rangle \approx&{} -\rho \theta^2 f_0^4 \left[-\frac{1}{2} + \frac{1}{2} + \frac{1}{2} - \frac{3}{4} - \frac{3}{4}\right] = \rho \theta^2 f_0^4.
\end{align}

To the lowest order in $\theta$, $f_0$, and $\rho$, the denominator of $\Lambda(f_0)$ can be found as 
\begin{align}
    \left\langle f_A^2f_B^2\cdot e^{-\frac{f_{A}+f_{B}}{f_0}}\right\rangle &\approx \theta^2 f_0^4 \partial_x^2 \partial_y^2 H_A H_B \Bigg\vert_{\substack{x=1 \\ y=1 \\ \tau=\infty}} = \theta^2 f_0^4 \frac{1}{x^2y^2}\Bigg\vert_{\substack{x=1 \\ y=1}} = \theta^2 f_0^4.
\end{align}

Thus, in the neutral limit for small $\rho$, $\Lambda(f_0) \approx \rho$.
\end{document}
